\documentclass{article}


\usepackage{CustomCommands}
\usepackage{PRIMEarxiv}

\usepackage[utf8]{inputenc} % allow utf-8 input
\usepackage[T1]{fontenc}    % use 8-bit T1 fonts
\usepackage{hyperref}       % hyperlinks
\usepackage{url}            % simple URL typesetting
\usepackage{booktabs}       % professional-quality tables
\usepackage{amsfonts}       % blackboard math symbols
\usepackage{nicefrac}       % compact symbols for 1/2, etc.
\usepackage{microtype}      % microtypography
\usepackage{lipsum}
\usepackage{fancyhdr}       % header
\usepackage{graphicx}       % graphics
\graphicspath{{media/}}     % organize your images and other figures under media/ folder

%Header
\pagestyle{fancy}
\thispagestyle{empty}
\rhead{ \textit{ }} 

% Update your Headers here
\fancyhead[LO]{Running Title for Header}
% \fancyhead[RE]{Firstauthor and Secondauthor} % Firstauthor et al. if more than 2 - must use \documentclass[twoside]{article}



  
%% Title
\title{A template for Arxiv Style
%%%% Cite as
%%%% Update your official citation here when published 
\thanks{\textit{\underline{Citation}}: 
\textbf{Authors. Title. Pages.... DOI:000000/11111.}} 
}

\author{
  Author1, Author2 \\
  Affiliation \\
  Univ \\
  City\\
  \texttt{\{Author1, Author2\}email@email} \\
  %% examples of more authors
   \And
  Author3 \\
  Affiliation \\
  Univ \\
  City\\
  \texttt{email@email} \\
  %% \AND
  %% Coauthor \\
  %% Affiliation \\
  %% Address \\
  %% \texttt{email} \\
  %% \And
  %% Coauthor \\
  %% Affiliation \\
  %% Address \\
  %% \texttt{email} \\
  %% \And
  %% Coauthor \\
  %% Affiliation \\
  %% Address \\
  %% \texttt{email} \\
}


\begin{document}
\maketitle


\begin{abstract}
\lipsum[1]
\end{abstract}


% keywords can be removed
\keywords{First keyword \and Second keyword \and More}


\section{Introduction}
\lipsum[2]
\lipsum[3]


\section{Background}
\label{sec:background}

\lipsum[4] See Section \ref{sec:background}.

\subsection{Small Set Flip}
\lipsum[5]
\begin{equation}
\xi _{ij}(t)=P(x_{t}=i,x_{t+1}=j|y,v,w;\theta)= {\frac {\alpha _{i}(t)a^{w_t}_{ij}\beta _{j}(t+1)b^{v_{t+1}}_{j}(y_{t+1})}{\sum _{i=1}^{N} \sum _{j=1}^{N} \alpha _{i}(t)a^{w_t}_{ij}\beta _{j}(t+1)b^{v_{t+1}}_{j}(y_{t+1})}}
\end{equation}

\subsubsection{Headings: third level}
\lipsum[6]

\paragraph{Paragraph}
\lipsum[7]

\section{Algorithm: Probabilistic Flip Method (PFM)}
\label{sec:algorithm}

\begin{algorithm}
	\caption{$\texttt{sort-top-K(T)}$}\label{alg:sort-top-k}
	\KwData{A vector $T \in \ZQubits$}
	\KwResult{A set $S$ of the top $K$ indices in $T$}
	\BlankLine
	$S \gets \texttt{indices of a descending radix sort of $T$'s rows}$\;
	\BlankLine
	\Return A set of the top $K$ indices in $S$\;
\end{algorithm}

% Main algorithm
\begin{algorithm}
	\caption{$\texttt{probabilistic-set-flip($E$)}$}\label{alg:pfm}
	\KwData{A syndrome $\sigma_0 \in \FSyndrome$}
	\KwResult{Deduced error $\DeducedE$ if the algorithm converges and $\bot$ otherwise}
	$\DeducedE \gets 0^{\NQubits}$\;
	$\sigma \gets \sigma_0$\;
		\While{$\exists F \in \GeneratorField: \OptNumerator > 0$}{
			$T \gets H_Z H_X^T \sigma$\;
			$\texttt{generators} \gets \texttt{sort-top-K}(T)$\;\label{alg:pfm:line:assign} 
			$\texttt{to-check} \gets \bigcup_{i \in \texttt{generators}} \mathcal{P}({\C_Z}_i)$\;
			$\correctionvec \gets \arg \max_{\correctionvec \in \texttt{to-check}} \frac{
				\OptNumerator
			}{
				|\correctionvec|
			}$\;
			$\DeducedE \gets \DeducedE \xor \correctionvec$\;
			$\sigma \gets \sigma \xor \sigma_X(\correctionvec)$\;	
		}
		\Return $\DeducedE$ if $|\sigma| = 0$, $\bot$ otherwise.
\end{algorithm}

\subsection{The Intuition}


\lipsum[4] See Section \ref{sec:background}.

\section{PFM Analysis}
\label{sec:analysis}

\lipsum[4] See Section \ref{sec:background}.

\section{PFM Numerical Simulations}
\label{sec:numerical-sim}

\lipsum[4] See Section \ref{sec:background}.

\section{PFM Future Outlook}
\label{sec:future}

\lipsum[4] See Section \ref{sec:background}.

\section{Conclusion}
\label{sec:conclusion}

\lipsum[4] See Section \ref{sec:background}.

\section{Acknowledgments}
\label{sec:ack}

\lipsum[4] See Section \ref{sec:background}.

\section{Examples of citations, figures, tables, references}
\label{sec:others}
\lipsum[8] \cite{kour2014real,kour2014fast} and see \cite{hadash2018estimate}.

The documentation for \verb+natbib+ may be found at
\begin{center}
  \url{http://mirrors.ctan.org/macros/latex/contrib/natbib/natnotes.pdf}
\end{center}
Of note is the command \verb+\citet+, which produces citations
appropriate for use in inline text.  For example,
\begin{verbatim}
   \citet{hasselmo} investigated\dots
\end{verbatim}
produces
\begin{quote}
  Hasselmo, et al.\ (1995) investigated\dots
\end{quote}

\begin{center}
  \url{https://www.ctan.org/pkg/booktabs}
\end{center}


\subsection{Figures}
\lipsum[10] 
See Figure \ref{fig:fig1}. Here is how you add footnotes. \footnote{Sample of the first footnote.}
\lipsum[11] 

\begin{figure}
  \centering
  \fbox{\rule[-.5cm]{4cm}{4cm} \rule[-.5cm]{4cm}{0cm}}
  \caption{Sample figure caption.}
  \label{fig:fig1}
\end{figure}

\subsection{Tables}
\lipsum[12]
See awesome Table~\ref{tab:table}.

\begin{table}
 \caption{Sample table title}
  \centering
  \begin{tabular}{lll}
    \toprule
    \multicolumn{2}{c}{Part}                   \\
    \cmidrule(r){1-2}
    Name     & Description     & Size ($\mu$m) \\
    \midrule
    Dendrite & Input terminal  & $\sim$100     \\
    Axon     & Output terminal & $\sim$10      \\
    Soma     & Cell body       & up to $10^6$  \\
    \bottomrule
  \end{tabular}
  \label{tab:table}
\end{table}

\subsection{Lists}
\begin{itemize}
\item Lorem ipsum dolor sit amet
\item consectetur adipiscing elit. 
\item Aliquam dignissim blandit est, in dictum tortor gravida eget. In ac rutrum magna.
\end{itemize}


\section{Conclusion}
Your conclusion here

\section*{Acknowledgments}
This was was supported in part by......

%Bibliography
\bibliographystyle{unsrt}  
\bibliography{references}  


\end{document}

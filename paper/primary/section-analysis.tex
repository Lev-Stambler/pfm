\newcommand{\lyingIndividual}{{L_j}}
\newcommand{\generatorScoreRV}{{G'_i}}
\newcommand{\lyingGenerator}{{E_i}}
\newcommand{\lyingGeneratorScore}{{E'_i}}
\newcommand{\sortIndex}{g_i}
\newcommand{\generatorScoreRVSorted}{{G'_{\sortIndex}}}
\newcommand{\syndromeRV}{S_j}
\newcommand{\generatorScoreOrderStat}{{G'_{(M - i)}}}
\newcommand{\generatorFlaggedStab}{{G_i}}
\newcommand{\generatorStabNeighbors}{{\aleph_i}}
\newcommand{\overcountGenerator}{\mathcal{O}^G_i}
\newcommand{\overcountError}{\mathcal{O}^E_i}
\newcommand{\PrS}{\half - \half\mleft(1-2p\mright)^\DegStab}
\newcommand{\PrL}{\half + \half \mleft( 1 - 2p\mright)^\DegStab - (1-p)^\DegStab}

\subsection*{Definitions}
Let us start by defining a slew of random variables
\begin{itemize}
	\item Let $\syndromeRV = 1$ if and only if $\vecSigma_j = 1$ and 0 otherwise.
	\item Let $\lyingIndividual = 1$ iff $\syndromeRV = 0 \wedge \exists k \in \Gamma(\text{stabilizer }j)$ 
	such that $\vecError_k = 1$ and 0 otherwise. In other words, think of $\lyingIndividual$ as indicating whether
	stabilizer $\lyingIndividual$ is ``lying" about not neighboring an error-ed bit.
	\item Let $\generatorStabNeighbors$ be the number of unique stabilizers in the neighborhood of a generator. More formally,
	$$
		\generatorStabNeighbors = \bigcup_{k \in \matrixGenerator_i} \Gamma(k).
	$$
	\item Let $\generatorFlaggedStab$ be the number of stabilizers in the neighborhood of a generator such 
	that the stabilizer is flagged. More formally,
	$$
		\generatorFlaggedStab = \sum_{j \in \generatorStabNeighbors} S_j
	$$
	\item Let $\generatorScoreRV$ be $T_i$ in the algorithm. Note that $\generatorScoreRV$ equals TODO: reference here
		$$
			\generatorScoreRV = \sum_{k \in \matrixGenerator_i} \sum_{j \in \Gamma(k)} \syndromeRV 
		$$
	\item Define $\lyingGenerator$ to be the number of ``lying" stabilizers in $\aleph$, or,
	$$
		\lyingGenerator = \sum_{j \in \generatorStabNeighbors} L_j
	$$
	\item $\lyingGeneratorScore$ be equal to the sum of the number of neighboring ``lying" stabilizers of each bit in generator $i$.
	Specifically,
	$$
		\lyingGeneratorScore = \sum_{k \in \matrixGenerator_i} \sum_{j \in \Gamma(k)} \lyingIndividual
	$$
	% \item g_1, g_2, ...., g_K relates to order g_1 = index of generator row where G'_g_i > ...
	% Then relate to order stat...
	% \item ORDERSTAT  ?? (think about)
	\item Define $\overcountGenerator = \generatorScoreRV - \generatorFlaggedStab$. $\overcountGenerator$ can be thought of as the
	number of ``over-counted" flagged stabilizers which contribute to $\generatorScoreRV$.
	\item Define $\overcountError = \lyingGeneratorScore - \lyingGenerator$. $\overcountError$ can be thought of as the
	number of ``over-counted" lying stabilizers which contribute to $\lyingGeneratorScore$.
\end{itemize}


\subsection*{Observations and Lemmas}

\begin{lemma}{For a generator, given $\lyingGenerator$ and $\generatorFlaggedStab$,
	we can find some correction vector $\correctionvec \in \F_2^\NQubits$
	such that \linebreak $\OptNumerator \geq \generatorFlaggedStab - \lyingGenerator$}
	and $|k| \leq \frac{1-\delta}{\DegBit} (\generatorFlaggedStab + \lyingGenerator)$ for $\generatorFlaggedStab > \lyingGenerator$.
\label{lemma:decrby}
	\begin{proof}
		TODO:	 part 1 is that there are 3 types of stabilizers in neighbourhood.
		Those from G', those from E, and those in neither. If those in neither, there is no
		neighbourhood in their error, so you can leave those bits alone. Then, by flipping bits
		connected to G' you decrease syndrome by G', but you add in at most E'
		
		part 2: each flipping bit effects at least (1-delta) $\DegBit$stables
	\end{proof}
\end{lemma}

\begin{lemma}{As $\NStablizer, \NBits \rightarrow \infty$ and for stabilizers $j_1, j_2, ..., j_C$ where $C$ is less than some constant,
		then $\Gamma(j_1), \Gamma(j_2), ..., \Gamma(j_C)$ are independent.
	}
	\label{lemma:indep-neighbor}
	\begin{proof}
		Let $A \subsetneq [C]$ and $B_{A_1}, B_{A_2}, ..., B_{A_{|A|}} \subseteq [N]$ such that $|B_{A_i}| = \DegStab$.	
		Then, to show independence, we want to show that for all $j \in C \setminus A$
		and some set $B_j \subseteq [N], |B_j| = \DegStab$,
		$$
			\Pr\mleft[\Gamma(j) = B_j\mright] = \Pr\mleft[\Gamma(j = B_j) \mid \Gamma(A_1) = B_{A_1}, ..., \Gamma(A_{|A|}) = B_{A_{|A|}})\mright].
		$$
		The above can easily be seen as $N, M \rightarrow \infty$ as
		$$
			\Pr_{B_j \subseteq [N], |B_j| = \DegStab}[\Gamma(j) = B_j] = \DegStab! \prod_{i =1}^{|B_j|} \Pr\mleft[{B_j}_i \in \Gamma(j) \mid {B_j}_{i-1}, ..., {B_j}_{1} \in \Gamma(j)\mright] = \DegStab! \cdot \frac{\DegStab!}{N^\DegStab}.
		$$

		Then, let $f(k, B_j, A)$ equal to the total number of instances some $k \in B_j$ is 
		in $B_{A_i}$ for all $i \in |A|$. Note that
		$$
			\Pr_{B_j \subseteq [N], |B_j| = \DegStab}[f(k, B_j, A) > 0] \rightarrow 0
		$$ as $N \rightarrow \infty$.
		Then, 
		\begin{align*}
			&\Pr_{B_j \subseteq [N], |B_j| = \DegStab}\mleft[\Gamma(j = B_j) \mid \Gamma(A_1) = B_{A_1}, ..., \Gamma(A_{|A|}) = B_{A_{|A|}})\mright] \\
				= &\DegStab! \prod_{i =1}^{|B_j|} \Pr\mleft[{B_j}_i \in \Gamma(j) \mid {B_j}_{i-1}, ..., {B_j}_{1} \in \Gamma(j), \Gamma(A_1) = B_{A_1}, ..., \Gamma(A_{|A|}) = B_{A_{|A|}})\mright]\\
				= &\DegStab! \prod_{i =1}^{|B_j|} \frac{(\DegStab + 1 - i) \mleft(\DegBit - f({B_j}_i, B_j, A)\mright)}{\DegBit N} \\
				\text{as } N \rightarrow \infty,\; &\DegStab! \cdot \frac{\DegStab!}{N^\DegStab}.
		\end{align*}
	\end{proof}
\end{lemma}

%% Lemma for S_j
\begin{lemma}{
	Assuming that the error rate is independent, then
	$S_1, S_2, ..., S_C$ are independent where $C \leq \DegStab \DegBit$ and
	$$
		\Pr[S_j = 1] = \PrS
	$$
}
\label{lemma:Sj}
\begin{proof}
	The $\Pr[S_j = 1]$ is then just equal to the probability that 
	$$
		\Pr\mleft[|\set{\vecError_k = 1 : k \in \Gamma(j)}| \texttt{ is odd}\mright].
	$$
	Note we are assuming by lemma \ref{lemma:indep-neighbor} that for $j \in [C]$, all the $\Gamma(j)$
	are independent. Thus, we have that 
	$
		\Pr\mleft[|\set{\vecError_k = 1 : k \in \Gamma(j)}| \texttt{ is odd}\mright]
	$ equals to the probability that a sample from $\binomial(\DegStab, p)$ is odd.
	This is then equal to $\PrS$. Note also that we can assume $S_1, S_2, ..., S_C$
	to be independent by lemma $\ref{lemma:indep-neighbor}$.
	
\end{proof}
\end{lemma}

\begin{lemma}{
	Assuming that the error rate is independent,
	$L_1, L_2, ..., L_C$ are independent for $C \leq \DegStab \DegBit$ and
	$$
		\Pr[L_j = 1] = \PrL.
	$$
}
\begin{proof}
	Let $s = |\set{\vecError_k = 1 : k \in \Gamma(j)}|$.
	Note that $L_j$ is $1$ iff $s > 0$ and $s$ is even.
	Then,
	$$
		\Pr[s > 0, s \text{ is even}] = \Pr[s\text{ is even}] - \Pr[s = 0] = \PrL.
	$$
	We can also take that $L_1, ..., L_C$ are independent by lemma $\ref{lemma:indep-neighbor}$.
\end{proof}
\end{lemma}


\subsection*{Distributions on the random variables}

\subsection*{Probability of single generator error}
% TODO: HERE IS MY THIRD STEP

\subsection*{Probability of error}

\subsection*{Error Probability Graphs}


\newpage
% \newcommand{\bitscorevec}{\boldsymbol{b}}
% \newcommand{\genscoreRVCorrected}{G}
% \newcommand{\bitscoreRV}{B}
% \newcommand{\generatorscorevec}{\boldsymbol{g}}
% \newcommand{\generatorscoreRV}{G'}
% \newcommand{\synd}{\boldsymbol{\boldsymbol{\sigma_X}}}
% \newcommand{\DegStable}{{\Delta_{\text{stablizer}}}}
% \newcommand{\DegBit}{{\Delta_{\text{bit}}}}
% \newcommand{\errorvec}{\boldsymbol{\boldsymbol{e}}}

% % TODO: say we are working with (a, b) regular graphs. 
% % TODO: say we are working with a $(\gamma, \delta)$ expander code

% The following section assumes that we are working with syndrome $\synd$,
% a generator matrix $H_Z$, and parity check matrix $H_X$.
% The analysis is the same for a syndrome, $\boldsymbol{\sigma_Z}$, generator matrix $H_X$, and
% parity check matrix $H_Z$.

% \subsection*{Notation}
% Given a vector $\boldsymbol{v}$, define $v_i$ to be the value of the $i$th row of $\boldsymbol{v}$.

% \subsection*{Definitions}
% Let $\DegStable$ equal to the degree of a stabilizer vertex. Note that due to the hypergraph's construction,
% all stabilizers have the same constant degree. 
% Let $\DegBit$ equal to the degree of a qubit vertex. As with the stabilizers, all qubits have the same constant degree.
% Also, for generator $k$, let $\aleph$ be the set of stabilizers neighboring the generator.
% Note that $|\aleph| \leq \DegStable \DegBit$.

% Given a syndrome, $\synd$, define a ``bit-score vector", $\bitscorevec = H_X^T \synd$ where
% $\bitscorevec \in \Z^{\NQubits}$. 
% Then, define a ``generator-score vector" as $\generatorscorevec = H_z \bitscorevec $ where
% $\generatorscorevec \in \Z^{\NStables}$. 
% Moreover, assume that for error $e \in \F_2^{\NQubits}$, $\Pr\mleft[e_i = 1\mright] = p$ for all $i \in [\NQubits]$ (i.e.\ 
% the error is modeled as independent). Let $q = 1-p$.
% Let $s_1, s_2, ..., s_{\NStables}$ denote the set of stabilizer vertices.
% Let $N_i = \sum_{j \in \Gamma(s_i)} \errorvec_j$ where $N_i \in \Z$. $N_i$ can be thought of
% as the number of qubits with an error in the neighborhood of stabilizer $i$.

% Also, let random variable $\syndromeRV_i \in F_2$ correspond  to $\synd_i$. Then we know that

% \newcommand{\SPr}{\half - \half\mleft(1-2p\mright)^\DegStable}
% \begin{align*}
% 	\Pr[S_i = 1] = \Pr\mleft[\texttt{$N_i$ is odd}\mright] = \SPr.
% \end{align*}

% Next, define indicator random variable, $\lyingIndividual_j$ to be $1$ if $\synd_i = 0$ and $N_i > 0$.
% Basically, $\lyingIndividual_j$ indicates whether a stabilizer check succeeds, but an error is in
% its neighborhood. I.e.\ stabilizer $j$ is ``lying."

% \newcommand{\LyingIndivPr}{\half + \half \mleft( 1 - 2p\mright)^\DegStable - (1-p)^\DegStable}

% % TODO: reference pr even (idk via some txtbook)
% So then,
% \begin{align*}
% 	\Pr[L_j = 1] &= \Pr\mleft[\synd_i = 0\; \middle|\; S_i > 0\mright] \\
% 		&= \Pr[\texttt{$S_i$ is even}] - \Pr[S_i = 0] \\
% 		&= \LyingIndivPr.
% \end{align*}


% Then define random variable, $\lyingGenerator_k$ to be
% $$
% 	\lyingGenerator_k = 
% 		\sum_{\texttt{Stabilizer $j$} \; \in \; \aleph_k}\;
% 				\lyingIndividual_j.
% $$

% $\lyingGenerator_k$ is basically the number of time a generator $k$, lies for all
% stabilizers neighboring the generator.

% We can then say that
% $$
% 	\lyingGenerator_k \sim \texttt{Binom}(\aleph_k, \LyingIndivPr).
% $$



% Then, let random variable $\bitscoreRV_i \in \Z$ correspond to $\bitscorevec_i$ and random variable $\generatorscoreRV_i \in \Z$
% correspond to $\generatorscorevec_i$. 


% So,
% $$
% 	\syndromeRV_i \sim \texttt{Bernoulli}\mleft(\SPr\mright).
% $$

% % TODO: NEXT STEP: TRELLO / WEEK IN ADVANCE
% Then,
% % TODO: better like neighbourhood thingy
% $$
% 	\bitscoreRV_i = \sum_{\texttt{Stabilizer $j$} \; \in \; \Gamma(\texttt{Bit $i$})} \syndromeRV_j
% $$
% So, 
% $$
% 	\bitscoreRV_i \sim \texttt{Binom}(\DegBit, \SPr).
% $$

% And then,
% $$
% 	\generatorscoreRV_k =
% 		\sum_{\texttt{Bit $i$} \; \in \; \Gamma(\texttt{generator } k)}\;
% 			\sum_{\texttt{Stabilizer } j \; \in \; \Gamma(\texttt{Bit } i)}
% 				\syndromeRV_j
% $$

% So then,
% $$
% 	\generatorscoreRV_k \sim \texttt{Binom}(\DegBit \DegStable, \SPr).
% $$

% The pfm algorithm (Algorithm \ref{alg:pfm}) on line \ref{alg:pfm:line:assign}
% gets the $K$ generators, indexed by $g_1, g_2, ..., g_K$, with the top values
% of $\generatorscorevec_i$ for $i \in [\NStables]$. WLOG, assume
% $\generatorscorevec_{g_1} \geq \generatorscorevec_{g_2} \geq ... \geq \generatorscorevec_{g_K}$.
% Then, we can think of $\E[\generatorscoreRV_{\OrderStatDenomi}]$ as the expected value of the $i$th
% top sample from $\NStables$ samples of the distribution defining $\generatorscoreRV_i$.

% Let random variable $\genscoreRVCorrected_k$ then equal
% $$
% 	\genscoreRVCorrected_k =
% 		\sum_{\texttt{Stabilizer $j$} \; \in \; \aleph_k} \syndromeRV_j.
% $$

% Note that $|\aleph| \geq (1 - \delta) \DegStable \DegBit$ because we are working with expander codes.
% So then, \begin{align*}
% 	\E[\genscoreRVCorrected_k] &= 
% 		\sum_{\texttt{Stabilizer $j$} \; \in \; \aleph_k} \E[\syndromeRV_j] \\
% 		&\geq (1 - \delta) \DegStable \DegBit \E_{j \in [\NStables]}[\syndromeRV_j]\\
% 		&\geq (1 - \delta) \E[\generatorscoreRV_k].
% \end{align*}

% % STEPS:

% % % Put into lemma
% % So then we can see that $$
% % 	\E[\lyingGenerator_k] = 
% % 		\sum_{\texttt{Stabilizer $j$} \; \in \; \aleph_k}\ \E[L_j] \leq \DegStable \DegBit \E[L_j]
% % 		= \DegStable \DegBit \mleft(\LyingIndivPr\mright).
% % % TODO: negative correlation argument!!!!!
% % $$

% \begin{lemma}{For a generator, given $\lyingGenerator_k$ and $\genscoreRVCorrected$,
% \label{lemma:decrby}
% 	we can find some correction vector $\correctionvec \in \F_2^\NQubits$
% 	such that \linebreak $\OptNumerator \geq \genscoreRVCorrected_k - \lyingGenerator_k$}
% 	and $|k| \leq \frac{1-\delta}{\DegBit} (\genscoreRVCorrected_k + \lyingGenerator_k)$ for $\genscoreRVCorrected > \lyingGenerator_k$.
% 	\begin{proof}
% 		TODO:	 part 1 is that there are 3 types of stabilizers in neighbourhood.
% 		Those from G', those from E, and those in neither. If those in neither, there is no
% 		neighbourhood in their error, so you can leave those bits alone. Then, by flipping bits
% 		connected to G' you decrease syndrome by G', but you add in at most E'
		
% 		part 2: each flipping bit effects at least (1-delta) $\DegBit$stables
% 	\end{proof}
% \end{lemma}


% So then for any $\errorvec$ where $|\errorvec| < \min(\gamma_An_A, \gamma_Bn_B)$ by TODO: cite hypergraph prod paper, we know that we can always successfully
% correct errors if we can find a $\correctionvec$ such that $\correctionvec$ is a subset of a generator and
% $$
% 	\frac{\OptNumerator}{\OptDenominator}	\geq \frac{1}{3}.
% $$

% \newcommand{\OptVarRV}{Q}
% \newcommand{\OptVarEqChangeable}[1]{
% \frac
% 			{\DegBit \mleft(#1_{\OrderStatDenomi} - \lyingGenerator_{\OrderStatDenomi}\mright)}
% 			{(1 - \delta)(#1_{\OrderStatDenomi} + \lyingGenerator_{g_k})}
% }
% \newcommand{\OptVarEq}{\OptVarEqChangeable{\genscoreRVCorrected}}

% \begin{lemma}{We claim the following holds for an $i \in [K]$ and for $p_S = \Pr[S_j = 1]$ for any stabilizer $j$}
	
% \begin{align*}
% 	&\Pr\mleft[\OptVarEq < \frac{1}{3}\mright] \\
% 	\leq &\sum_{e = 0}^{\DegBit - 1} 
% 			\OrderP\mleft(\DegBit\DegStable - \DegStable e, p_S, i, \frac{(3\DegBit + 1 - \delta) e}{3\DegBit -1 + \delta}\mright) \cdot \Pr[E_{\OrderStatDenomi} = e] 
% 			+ \sum_{e = \DegBit}^{\DegBit\DegStable}\Pr[E_{\OrderStatDenomi} = e]
% \end{align*}
% where
% $$
% 	\OrderP(n, p, i, v) = \Pr\mleft[W_i \leq v \mright]
% $$
% and $W_i$ is the $i$th largest order statistic from $M$ samples of $\texttt{Binomial}(n, p)$.

% See appendix TODO: cite for details
% \end{lemma}

% So then, \begin{align*}
% 	\Pr\mleft[\texttt{loop cannot find a correcting error}\mright] &\leq
% 		\prod_{i \in [K]} \Pr\mleft[\OptVarEq < \frac{1}{3} \mright]
% \end{align*}

% % Hmmmmm.... I think just run simulations on the above ^^ (j draw from G', E...)
% % Let the random variable $Q(K)$ then equal $\OptVarEq$.

% % So then, for a given $K$,
% % $$
% % 	\Pr\mleft[Q(K) \geq \frac{1}{3}\mright] = 1 - \Pr[Q(K) \leq \frac{1}{3}] \geq 1 - \frac{}{}
% % $$

%  lemma \ref{lemma:decrby}, 


% % Then, assuming that $p < 0.5$, $\Pr\mleft[\synd_i = 0\; \middle|\; S_i > 0\mright] < \half$

\newcommand{\bitscorevec}{\boldsymbol{b}}
\newcommand{\lyingIndividual}{L}
\newcommand{\lyingGenerator}{E}
\newcommand{\genscoreRVCorrected}{G'}
\newcommand{\bitscoreRV}{B}
\newcommand{\syndromeRV}{S}
\newcommand{\generatorscorevec}{\boldsymbol{g}}
\newcommand{\generatorscoreRV}{G}
\newcommand{\synd}{\boldsymbol{\boldsymbol{\sigma_X}}}
\newcommand{\NTQ}{{\Delta_{\text{stablizer}}}}
\newcommand{\NTS}{{\Delta_{\text{bit}}}}
\newcommand{\errorvec}{\boldsymbol{\boldsymbol{e}}}

% TODO: say we are working with (a, b) regular graphs. 
% TODO: say we are working with a $(\gamma, \delta)$ expander code

The following section assumes that we are working with syndrome $\synd$,
a generator matrix $H_Z$, and parity check matrix $H_X$.
The analysis is the same for a syndrome, $\boldsymbol{\sigma_Z}$, generator matrix $H_X$, and
parity check matrix $H_Z$.

\subsection*{Notation}
Given a vector $\boldsymbol{v}$, define $v_i$ to be the value of the $i$th row of $\boldsymbol{v}$.

\subsection*{Definitions}
Let $\NTQ$ equal to the degree of a stabilizer vertex. Note that due to the hypergraph's construction,
all stabilizers have the same constant degree. 
Let $\NTS$ equal to the degree of a qubit vertex. As with the stabilizers, all qubits have the same constant degree.
Also, for generator $k$, let $\aleph$ be the set of stabilizers neighboring the generator.
Note that $|\aleph| \leq \NTQ \NTS$.

Given a syndrome, $\synd$, define a ``bit-score vector", $\bitscorevec = H_X^T \synd$ where
$\bitscorevec \in \Z^{\NQubits}$. 
Then, define a ``generator-score vector" as $\generatorscorevec = H_z \bitscorevec $ where
$\generatorscorevec \in \Z^{\NStables}$. 
Moreover, assume that for error $e \in \F_2^{\NQubits}$, $\Pr\mleft[e_i = 1\mright] = p$ for all $i \in [\NQubits]$ (i.e.\ 
the error is modeled as independent). Let $q = 1-p$.
Let $s_1, s_2, ..., s_{\NStables}$ denote the set of stabilizer vertices.
Let $N_i = \sum_{j \in \Gamma(s_i)} \errorvec_j$ where $N_i \in \Z$. $N_i$ can be thought of
as the number of qubits with an error in the neighborhood of stabilizer $i$.

Then, let random variable $\bitscoreRV_i \in \Z$ correspond to $\bitscorevec_i$ and random variable $\generatorscoreRV_i \in \Z$
correspond to $\generatorscorevec_i$. Also, let random variable $\syndromeRV_i \in F_2$ correspond  to $\synd_i$.

\newcommand{\SPr}{\half - \half\mleft(1-2p\mright)^\NTQ}
\begin{align*}
	\Pr[S_i = 1] = \Pr\mleft[\texttt{$N_i$ is odd}\mright] = \SPr
\end{align*}

So,
$$
	\syndromeRV_i \sim \texttt{Bernoulli}\mleft(\SPr\mright).
$$

% TODO: NEXT STEP: TRELLO / WEEK IN ADVANCE
Then,
% TODO: better like neighbourhood thingy
$$
	\bitscoreRV_i = \sum_{\texttt{Stabilizer $j$} \; \in \; \Gamma(\texttt{Bit $i$})} \syndromeRV_j
$$
So, 
$$
	\bitscoreRV_i \sim \texttt{Binom}(\NTS, \SPr).
$$

And then,
$$
	\generatorscoreRV_k =
		\sum_{\texttt{Bit $i$} \; \in \; \Gamma(\texttt{generator } k)}\;
			\sum_{\texttt{Stabilizer } j \; \in \; \Gamma(\texttt{Bit } i)}
				\syndromeRV_j
$$

So then,
$$
	\generatorscoreRV_k \sim \texttt{Binom}(\NTS \NTQ, \SPr).
$$

The pfm algorithm (Algorithm \ref{alg:pfm}) on line \ref{alg:pfm:line:assign}
gets the $K$ generators, indexed by $g_1, g_2, ..., g_K$, with the top values
of $\generatorscorevec_i$ for $i \in [\NStables]$. WLOG, assume
$\generatorscorevec_{g_1} \geq \generatorscorevec_{g_2} \geq ... \geq \generatorscorevec_{g_K}$.
Then, we can think of $\E[\generatorscoreRV_{g_i}]$ as the expected value of the $i$th
top sample from $\NStables$ samples of the distribution defining $\generatorscoreRV_i$.

Let random variable $\genscoreRVCorrected_k$ then equal
$$
	\genscoreRVCorrected_k =
		\sum_{\texttt{Stabilizer $j$} \; \in \; \aleph_k} \syndromeRV_j.
$$

Note that $|\aleph| \geq (1 - \delta) \NTQ \NTS$ because we are working with expander codes.
So then, \begin{align*}
	\E[\genscoreRVCorrected_k] &= 
		\sum_{\texttt{Stabilizer $j$} \; \in \; \aleph_k} \E[\syndromeRV_j] \\
		&\geq (1 - \delta) \NTQ \NTS \E_{j \in [\NStables]}[\syndromeRV_j]\\
		&\geq (1 - \delta) \E[\generatorscoreRV_k].
\end{align*}

% Then, let $\genscoreRVCorrected$ be the normalized version of $\generatorscoreRV$ where
% $$
% 	\genscoreRVCorrected = \frac{\generatorscoreRV - \mu}{\sigma}
% $$

% \newcommand{\muexp}{\NTS \NTQ \mleft(\SPr\mright)}
% \newcommand{\sigmaexp}{ \sqrt{\NTS \NTQ (\SPr) (1 - (\SPr))}}
% where $\mu = \muexp$ and $\sigma = \sigmaexp$.

% For a standard Gaussian random variable, $Z$, we then know that by the BLANK BLANK BOUND TODO:,
% $$
% 	|\Pr[\genscoreRVCorrected \geq u] - \Pr[Z \geq u]| \leq 0.56 \cdot (\beta)
% $$
% where
% % Hmmm... not sure about Beta bound
% % Hmmm.. about equal
% % TODO: NORMALIZE, put into appendix the proof
% $$
% 	\beta =
% 		\sum_{i=1}^{\NTS \NTQ} \E\mleft[|B'_i|^3\mright] \leq \frac{0.56}{\sqrt{\NTS \NTQ}}
% $$

% 	% &\leq \sum_{i=1}^{\NTS \NTQ} \E\mleft[\mleft|\frac{1}{\sigma}\mright|^3\mright] \\
% 	% &= \frac{\NTS \NTQ}{\sigmaexp^3}\\
% 	% &= \frac{\NTS \NTQ}{
% 	% 	\sqrt{
% 	% 		\NTS \NTQ \mleft(\frac{1}{4} - \frac{\mleft(1-2p\mright)^\NTQ}{4}\mright)
% 	% 	}
% 	% }.
% % TODO: SMTHNG WRONG HERE

% So then, 
% $
% 	\Pr[\genscoreRVCorrected \geq u] \leq O(\beta) + \Pr[Z \geq u]
% $,
% and by TODO: we know that $\Pr[Z \geq u] \leq \frac{\phi(u)}{u}$. Combining the two equations,
% we get that 
% $$
% 	\Pr[\genscoreRVCorrected \geq u] \leq O(\beta) + \frac{\phi(u)}{u}.
% $$
% For $i \in [\NStables]$, if we then fix $\Pr[\genscoreRVCorrected \geq u] = \frac{i}{\NStables}$
% and solve for $u$ we can get the expected value of $u$ such that $\genscoreRVCorrected$ is larger than
% or equal to the value the top $i$ samples from $\NStables$.
% So then, if we solve for $u$ such that $O(\beta) + \frac{\phi(u)}{u} = \frac{i}{\NStables}$,
% we can get a lower bound on $u$ for the expected value of the top $i$th sample.
% Using Sage, we get 
% $$
% u \geq \frac{\sqrt{{\NTQ} {\NTS}} M \cdot e^{\left(-\frac{1}{2} \, u^{2}\right)}}{\sqrt{2 \pi {\NTQ} {\NTS}} K - 14 \, \sqrt{2} \sqrt{\pi} M}.
% $$
% \newcommand{\uNonLn}{\frac{\sqrt{{\NTQ} {\NTS}} M}{\sqrt{2 \pi {\NTQ} {\NTS}} K - 14 \, \sqrt{2} \sqrt{\pi} M}}

% So then,
% \begin{align*}
% 	\frac{u}{\exp(-\half u^2)} &\geq \uNonLn\\
% 	\Rightarrow \half u^2 + \ln u &\geq \ln \uNonLn \\
% 	\Rightarrow u &\geq \sqrt{2 \ln \uNonLn} \cdot (1 - \epsilon) \\
% \end{align*}

% for some sufficiently large $u$ and $\epsilon > 0$.

% Then, get that $u >= SMTHNG $ or something for SMTHNG SMTHNG.


% where $\genscoreRVCorrected$ is a random variable and $\genscoreRVCorrected \in Z$.

% Note that due to the expander graph property, generator $k$ has at least 
% $(1 - \delta)\NTQ$ unique stabilizer vertices in its neighborhood.


% STEPS:
Next, define indicator random variable, $\lyingIndividual_j$ to be $1$ if $\synd_i = 0$ and $N_i > 0$.
Basically, $\lyingIndividual_j$ indicates whether a stabilizer check succeeds, but an error is in
its neighborhood. I.e.\ stabilizer $j$ is ``lying."

\newcommand{\LyingIndivPr}{\half + \half \mleft( 1 - 2p\mright)^\NTQ - (1-p)^\NTQ}

% TODO: reference pr even (idk via some txtbook)
So then,
\begin{align*}
	\Pr[L_j = 1] &= \Pr\mleft[\synd_i = 0\; \middle|\; S_i > 0\mright] \\
		&= \Pr[\texttt{$S_i$ is even}] - \Pr[S_i = 0] \\
		&= \LyingIndivPr
\end{align*}


Then define random variable, $\lyingGenerator_k$ to be
$$
	\lyingGenerator_k = 
		\sum_{\texttt{Stabilizer $j$} \; \in \; \aleph}\;
				\lyingIndividual_j.
$$

$\lyingGenerator_k$ is basically the number of time a generator $k$, lies for all
stabilizers neighboring the generator.

We can then say that
$$
	\lyingGenerator_k \sim \texttt{Binom}(\NTS \NTQ, \LyingIndivPr).
$$
% % Put into lemma
% So then we can see that $$
% 	\E[\lyingGenerator_k] = 
% 		\sum_{\texttt{Stabilizer $j$} \; \in \; \aleph_k}\ \E[L_j] \leq \NTQ \NTS \E[L_j]
% 		= \NTQ \NTS \mleft(\LyingIndivPr\mright).
% % TODO: negative correlation argument!!!!!
% $$

\begin{lemma}{For a generator, given $\lyingGenerator_k$ and $\genscoreRVCorrected$,
\label{lemma:decrby}
	we can find some correction vector $\correctionvec \in \F_2^\NQubits$
	such that \linebreak $\OptNumerator \geq \genscoreRVCorrected_k - \lyingGenerator_k$}
	and $|k| \leq \frac{1-\delta}{\NTS} (\genscoreRVCorrected_k + \lyingGenerator_k)$ for $\genscoreRVCorrected > \lyingGenerator_k$.
	\begin{proof}
		TODO:	 part 1 is that there are 3 types of stabilizers in neighbourhood.
		Those from G', those from E, and those in neither. If those in neither, there is no
		neighbourhood in their error, so you can leave those bits alone. Then, by flipping bits
		connected to G' you decrease syndrome by G', but you add in at most E'
		
		part 2: each flipping bit effects at least (1-delta) $\NTS$stables
	\end{proof}
\end{lemma}


So then, by TODO: cite hypergraph prod paper, we know that we can always successfully
correct errors if, for each round of the while loop,
$$
	\frac{\OptNumerator}{\OptDenominator}	\geq \frac{1}{3}.
$$

\newcommand{\OptVarRV}{Q}
\newcommand{\OptVarEq}{
\frac
			{\NTS \mleft(\genscoreRVCorrected_{g_i} - \lyingGenerator_{g_i}\mright)}
			{(1 - \delta)(\genscoreRVCorrected_{g_i} + \lyingGenerator_{g_k})}
			\geq \frac{1}{3}
}
So then, \begin{align*}
	\Pr\mleft[\texttt{loop cannot find a correcting error}\mright] &\leq
		\prod_{i \in [K]} \mleft(1 - \Pr\mleft[\OptVarEq\mright]\mright)
\end{align*}

% Hmmmmm.... I think just run simulations on the above ^^ (j draw from G', E...)
% Let the random variable $Q(K)$ then equal $\OptVarEq$.

% So then, for a given $K$,
% $$
% 	\Pr\mleft[Q(K) \geq \frac{1}{3}\mright] = 1 - \Pr[Q(K) \leq \frac{1}{3}] \geq 1 - \frac{}{}
% $$

 lemma \ref{lemma:decrby}, 


% Then, assuming that $p < 0.5$, $\Pr\mleft[\synd_i = 0\; \middle|\; S_i > 0\mright] < \half$

\begin{algorithm}
	\caption{$\texttt{sort-top-K(T)}$}\label{alg:sort-top-k}
	\KwData{A vector $T \in \ZQubits$}
	\KwResult{A set $S$ of the top $K$ indices in $T$}
	\BlankLine
	$S \gets \texttt{indices of a descending radix sort of $T$'s rows}$\;
	\BlankLine
	\Return A set of the top $K$ indices in $S$\;
\end{algorithm}

% Main algorithm
\begin{algorithm}
	\caption{$\texttt{probabilistic-set-flip($E$)}$}\label{alg:pfm}
	\KwData{A syndrome $\sigma_0 \in \FSyndrome$}
	\KwResult{Deduced error $\DeducedE$ if the algorithm converges and $\bot$ otherwise}
	$\DeducedE \gets 0^{\NQubits}$\;
	$\sigma \gets \sigma_0$\;
		\While{$\exists F \in \GeneratorField: \OptNumerator > 0$}{
			$T \gets H_Z H_X^T \sigma$\;
			$\texttt{generators} \gets \texttt{sort-top-K}(T)$\;\label{alg:pfm:line:assign} 
			$\texttt{to-check} \gets \bigcup_{i \in \texttt{generators}} \mathcal{P}({\C_Z}_i)$\;
			$\correctionvec \gets \arg \max_{\correctionvec \in \texttt{to-check}} \frac{
				\OptNumerator
			}{
				|\correctionvec|
			}$\;
			$\DeducedE \gets \DeducedE \xor \correctionvec$\;
			$\sigma \gets \sigma \xor \sigma_X(\correctionvec)$\;	
		}
		\Return $\DeducedE$ if $|\sigma| = 0$, $\bot$ otherwise.
\end{algorithm}

\subsection{A Moral Reason/ Intuition}
The algorithm is essentially the same as Small Set Bit Flip [TODO: CITE] with a minor
difference, only a constant number of generators are checked.
The idea here is that given a syndrome $\sigma$ and a parity check matrix, $H_X$,
the $i$th row of $H_X^T \sigma$ equals the number of error-ed checks that a
qubit touches. Then, the $k$th row of $H_ZH_X^T\sigma$ is roughly correlated to
the number of error-ed checks that the qubits in the $k$th generator touch.
This rough correlation comes from the fact that we are working with expander codes.
So then, if you get the generators touching the most error-ed stabilizers, 
it would stand to reason that flipping some subset of qubits from a ``highly error-ed generator" would
result in decreasing the syndrome.
